\documentclass{beamer}
\usepackage{amsfonts,amsmath,oldgerm}
\usepackage[nameinlink]{cleveref}
\usepackage{graphicx}
\usepackage{hyperref}
\usepackage[
  backend=biber,
  style=alphabetic,
  sorting=anyt,
  minnames=3,
  minalphanames=3
]{biblatex}

\hypersetup{
    colorlinks=true,
    citecolor=red,
    linkcolor=white,
    urlcolor=black,
    pdftitle={MLCS Seminar},
}

\usetheme{sintef}

\newcommand{\N}{\mathbb{N}}                     % Natural Numbers
\newcommand{\curlyquotes}[1]{\textquotedblleft #1\textquotedblright}

\usefonttheme[onlymath]{serif}
\setbeamercovered{transparent}

\titlebackground*{assets/background}

\title{Turing Degrees and the Friedberg-Muchnik Theorem}
\subtitle{Mathematical Logic for Computer Science}
\course{Master's Degree in Computer Science}
\author{\href{https://github.com/Exyss/}{Simone Bianco}}
\IDnumber{1986936}
\date{Academic Year 2024/2025}

\addbibresource{../references.bib}

\begin{document}
\maketitle


% \begin{frame}

% This template is a based on \hrefcol{https://www.overleaf.com/latex/templates/sintef-presentation/jhbhdffczpnx}{SINTEF Presentation} from \hrefcol{mailto:federico.zenith@sintef.no}{Federico Zenith} and its derivation \hrefcol{https://github.com/TOB-KNPOB/Beamer-LaTeX-Themes}{Beamer-LaTeX-Themes} from Liu Qilong

% \vspace{\baselineskip}

% In the following you find a brief introduction on how to use \LaTeX\ and the beamer package to prepare slides, based on the one written by \hrefcol{mailto:federico.zenith@sintef.no}{Federico Zenith} for \hrefcol{https://www.overleaf.com/latex/templates/sintef-presentation/jhbhdffczpnx}{SINTEF Presentation}

% % This template is released under \hrefcol{https://creativecommons.org/licenses/by-nc/4.0/legalcode}{Creative Commons CC BY 4.0} license

% \end{frame}

\section{Introduction}

\begin{frame}{Introduction}
\framesubtitle{Notation}

    $\Phi_{i,s}^A(x)$ denotes the \textbf{computation} of the TM $M_i$ for $s \in \N$ steps on input $x \in \N$ while having access to an oracle for the subset $A \subseteq \N$

    \uncover<+->{
        \begin{itemize}[<+->]
        \item $\Phi_{i,s}^A(x) \downarrow$ : the computation halts after $s$ steps
        \item $\Phi_{i}^A(x) \downarrow$ : there is a $s \in \N$ such that $\Phi_{i,s}^A(x) \downarrow$
        \item $\Phi_{i}^A(x) \uparrow$ : there is no $s \in \N$ such that $\Phi_{i,s}^A(x) \downarrow$
        \item $\phi_{i}^A(x)$ : output of the computation (defined iff $\Phi_{i}^A(x) \downarrow$)
        \end{itemize}
    }
\end{frame}

\begin{frame}{Introduction}
\framesubtitle{Definitions}
    Given a set $S \subseteq \N$, we say that $S$ is:
    \begin{itemize}[<+->]
        \item \textbf{Semi-decidable} if $\exists i \in \N$ such that $\forall x \in S$ it holds that $\Phi_i(x) \downarrow$ and $\phi_i(x) = 1$.
        \item \textbf{Decidable} if $\exists i \in \N$ such that $\forall x \in \N$ it holds that $\Phi_i(x) \downarrow$ and $\phi_i(x) = 1$ if $x \in S$, otherwise $\phi_i(x) = 0$.
        \item \textbf{Recursively enumerable} if there is an algorithmic procedure $\mathcal{A} : \N \to \{0,1\}$ such that $S = \{A(0), A(1), A(2), \ldots\}$
    \end{itemize}

    \quad

    \uncover<+-> {
        \textbf{Obs. 1:} $S$ is semi-decidable if and only if it is r.e.

        \textbf{Obs. 2:} $S$ is decidable if and only if both $S$ and $\overline{S}$ are semi-decidable.
    }
\end{frame}

\begin{frame}{Introduction}
\framesubtitle{Turing's work}
    \textcite{turing} proved that:
    \begin{itemize}[<+->]
        \item  Some sets that are semi-decidable but undecidable (e.g. $H = \{(i,x) \mid \Phi_i(x) \downarrow\}$).
        
        \item Some sets cannot be semi-decided (e.g. $\overline{H}$).
    \end{itemize}

    \uncover<+-> {
        This gives three degrees of computability: \textbf{solvable} problems, \textbf{semi-solvable} problems and \textbf{unsolvable} problems.
    }

    \quad

    \uncover<+-> {
        Are there some other degrees of computability?
    }
\end{frame}

\section{Degrees of Unsolvability}

\begin{frame}{Degrees of Unsolvability}
\framesubtitle{Turing degrees}
    \textcite{post_degrees} formalized the idea of computability degrees through \textbf{Turing reductions}.

    \begin{itemize}[<+->]
        \item \textit{Turing reducibility}: $A \leq_T B$ when $\exists i \in \N$ such that $\Phi_i^B(x) \downarrow$ for all $x \in \N$ and $\phi_i^B = A$.
        \item \textit{Turing equivalence}: $A \equiv_T B$ when $A \equiv_T B$, when $A \leq_T B$ and $B \leq_T A$
        \item The set $\mathcal{D} = 2^{\N}/_{\equiv_T}$ is referred to as the set of \textbf{Turing degrees}.
    \end{itemize}
\end{frame}

\begin{frame}{Degrees of Unsolvability}
\framesubtitle{Turing degrees}

    \begin{itemize}[<+->]
        \setlength{\itemindent}{-2em}

        \item[] We say that $[A]$ is \textbf{lower} than $[B]$, written as $[A] \preceq [B]$, if $A \leq_T B$.
        
        \quad
        
        \item[] $\preceq$ is a (partial) order over the set $\mathcal{D}$, forming an hierarchy of unsolvability degrees.
        
        \quad

        \item[] \textbf{Prop. 1}: There is an unique degree containing all the decidable problems
    
        \begin{itemize}
            \item This unique class is referred to as the $0$ degree (formally $0 = [\varnothing]$)
        \end{itemize}

        \quad

        \item[ ]\textbf{Prop. 2}: There is no degree below $0$
        \begin{itemize}
            \item If $[A] \preceq 0$ then $A$ is decidable, thus $[A] = 0$
        \end{itemize}
    \end{itemize}
\end{frame}

\begin{frame}{Degrees of Unsolvability}
\framesubtitle{The jump operator}
    The strict relation $[A] \prec [B]$ between degrees can be easily forced through \textbf{Turing jumps}.
    \begin{itemize}[<+->]
        \item Given a set $X \subseteq \N$, the \textit{Turing jump} of $X$ is the set $X' = \{i \mid \Phi_i^X(i) \downarrow\}$
        \begin{itemize}
            \item $X <_T X'$ since $X'$ is obtained by forcing a variant of the Halting problem on TMs with $X$ as an oracle
        \end{itemize}
        \item \textbf{Obs.:} The jump $0'$ of $0$ is exactly the class containing the Halting problem
        \[\varnothing' = \{i \mid \Phi_i^\varnothing(i) \downarrow\} = \{i \mid \Phi_i(i) \downarrow\} \equiv_T H\]
    \end{itemize}
\end{frame}

\begin{frame}{Degrees of Unsolvability}
\framesubtitle{The jump operator}
    \begin{itemize}[<+->]
        \item \textbf{Thm.:} $[A] \leq_m H$ if and only if $A$ is r.e.
        
        \begin{itemize}
            \item If $A \leq_m H$ then $A$ is trivially r.e. since $H$ is r.e.
            \item If $A$ is r.e. then it has a semi-decider $M_i$. Let $M_j$ be a new semi-decider such that $\Phi_j(x) \downarrow$ if $\phi_i(x) = 1$, otherwise $\Phi_j(x) \uparrow$.
        \end{itemize}

        \item \textbf{Cor.:} Every degree containing a r.e. set is below $0'$
        \item \textbf{Obs.}: If a degree contains a r.e. set, the other sets \underline{aren't} forced to also be r.e.
            \begin{itemize}
                \item E.g.: $\overline{H} \in 0'$, but  $\overline{H}$ is not r.e 
            \end{itemize}
    \end{itemize}
\end{frame}

\section{Post's problem}

\begin{frame}{Post's problem}
\framesubtitle{The main question}
    \begin{itemize}[<+->]
        \item The jump operator can be iteratively applied to get infinite levels of the hierarchy, i.e. $0 \prec 0' \prec 0'' \prec \ldots$. This makes \curlyquotes{going upwards} not so interesting.
        \item \textcite{post_degrees} proved that for each degree $A$ there is another degree $B$ that is incomparable with $A$.
        \begin{itemize}
            \item \textbf{Cor.:} There is at least one Turing degree that is incomparable with both $0$ and $0'$ 
        \end{itemize}

        \item \textbf{Post's problem}: is there a degree $d$ such that $0 \prec d \prec 0'$?
    \end{itemize}
\end{frame}

\begin{frame}{Post's problem}
\framesubtitle{The main question}
    \begin{itemize}[<+->]
        \item Post's problem was solved 12 years later independently by \textcite{friedberg, muchnik} through the \textbf{finite injury priority method}.
        \item The method is an improvement on the \textbf{finite extension method} developed by Post in his original works
        \item Since the FIP method involves constructions that are way more complex than those of the FE method, we'll first give an example of the latter
    \end{itemize}
\end{frame}

\begin{frame}{The finite extension method}
\framesubtitle{The setup}

    \begin{itemize}[<+->]
        \item Given a set $A$, we want to define a countable list of \textbf{requirements} $\{R_i\}_{i \in \N}$, each to be satisfied by a string.
        
        \item Strings are to be considered as an infinite tape of cells, each marked by an index and containing either a $0$, a $1$ or an undefined value.
        
        \begin{itemize}
            \item In some sense, each string can be viewed as a partial function on $\{0,1\}$.
        \end{itemize}
        
        \item We start with the empty string and on each step $s \in \N$, we construct a new finite string $A_{s+1}$ that extends $A_s$ and satisfies $R_0, R_1, \ldots, R_{s+1}$.
    \end{itemize}
\end{frame}

\begin{frame}{The finite extension method}
\framesubtitle{The Kleene-Post theorem}
    \cite{kleene_post} \textbf{Thm.:} There are two sets $A$ and $B$ that are incomparable
    \begin{itemize}[<+->]
        \item For each $i \in \N$, we define the requirement $R_{2i}$ as $\phi_i^A \neq B$ and $R_{2i+1}$ as $\phi_{i}^B \neq A$
        \begin{itemize}
            \item When $\Phi_i^A(x) \uparrow$ or $\Phi_i^B(x) \uparrow$, the requirements $R_{2i}, R_{2i+1}$ are considered to be satisfied.
        \end{itemize} 
        \item Let $\{R_i\}_{i \in \N}$ be the list of requirements
        \item Let $A_0 = B_0 = \varepsilon$
    \end{itemize}
\end{frame}

\begin{frame}{The finite extension method}
\framesubtitle{The Kleene-Post theorem}
    \begin{itemize}[<+->]
        \item Consider a generic step $s \in \N$
        \begin{itemize}
            \item We assume that $s = 2i$ since the case $2i+1$ is symmetrically constructed
        \end{itemize}
        \item Choose any index $x \in \N$ such that $B_s(x) = *$
        \begin{itemize}
            \item Guaranteed to exist!
        \end{itemize}
        \item If there is any finite extension $A'$ of $A_s$ such that $\Phi_i^{A'}(x) \downarrow$ then we set:
        \[A_{s+1} = A' \qquad B_{s+1}(y) = \left \{ \begin{array}{ll}
            B_s(y) & \text{if } y \neq x \\
            1-\phi_{i}^{A'}(x) & \text{if } y = x
        \end{array}\right .\]
        \item The requirement $R_{2i}$ is satisfied by this flipped bit.
    \end{itemize}
\end{frame}

\begin{frame}{The finite extension method}
\framesubtitle{The Kleene-Post theorem}
    \begin{itemize}[<+->]
        \item If no such finite extension $A'$ of $A_s$ exists, then $\Phi_i^{A'}(x) \uparrow$ for all $A'$.
        \item Hence, we can trivially set
        \[A_{s+1} = A_{s} \qquad B_{s+1}(y) = \left \{ \begin{array}{ll}
            B_s(y) & \text{if } y \neq x \\
            \mathrm{rand}(\{0,1\}) & \text{if } y = x
        \end{array}\right .\]
        \item The requirement $R_{2i}$ is automatically satisfied since $\Phi_i^{A_{s+1}}(x) \uparrow$.
        \item $A = \bigcup\limits_{i \in \N} A_i$ and $B = \bigcup\limits_{i \in \N} B_i$ are such that $\Phi_s^A(x) \neq B$ and $\Phi_s^B(x) \neq A$ for all $s \in \N$
    \end{itemize}
    $\hfill \square$
\end{frame}

\begin{frame}{The finite extension method}
\framesubtitle{Strenghts and weaknesses}
    
    \begin{itemize}[<+->]
        \item We observe that this general method can be easily modified and extended.
        \begin{itemize}
            \item E.g.: we can force that $[A], [B] \leq 0'$ or even $[A'], [B'] \leq 0'$
        \end{itemize}
        \item However, the construction used can never guarantee the recursive enumerability of the two sets.
        \item This problem is fixed by the \textbf{finite injury priority method}
    \end{itemize}

\end{frame}

\begin{frame}{The finite injury priority method}
\framesubtitle{The idea}
    \begin{itemize}[<+->]
        \item \textbf{Injury}: a requirement gets injured when its satisfaction is not guaranteed anymore.
        
        \item The key idea is to assign a priority level to each requirement and force two conditions during the construction of the set $A$:
    \begin{enumerate}
        \item Each requirement has finitely many requirements with higher priority
        \item Each requirement can be injured only by requirements with higher priority
    \end{enumerate}
    \end{itemize}

\end{frame}

\begin{frame}{The finite injury priority method}
\framesubtitle{The query function}
    \begin{itemize}[<+->]
        \item To keep track of injuries, we'll use the following \textbf{query function}
        
        \item Let $A \subseteq \N$ and let $i,s,x \in \N$. The query function $\omega_{i,s}^A$ is defined as:
        \[\omega_{i,s}^A(x) = \left \{ \begin{array}{ll}
            \max \{z \in \N \mid A(z) \text{ is queried in } \Phi_{i,s}^A\} & \text{if } \Phi_{i,s}^A(x) \downarrow \\
            -1 & \text{otherwise}
        \end{array}\right .\]
        
    \end{itemize}

\end{frame}

\begin{frame}{The finite injury priority method}
\framesubtitle{The Friedberg-Muchnik theorem}
    \cite{friedberg,muchnik} \textbf{Thm.:} There are two r.e. sets $A$ and $B$ that are incomparable
    \begin{itemize}[<+->]
        \item The requirements are defined as in the previous theorem
        \item $A_0, A_1, A_2, \ldots$ and $B_0, B_1, B_2, \ldots$ are now \textbf{sets}
        \item We say that an index $x$ is a \textbf{witness} for the requirement $R_{2i}$ at step $s$ if $\Phi_{i,s}^{B_s}(x) \downarrow$ and $\phi_{i,s}^{B_s}(x) \neq A_s(x)$ (symmetric definition for $R_{2i+1}$)

    \end{itemize}
\end{frame}

\begin{frame}{The finite injury priority method}
\framesubtitle{The Friedberg-Muchnik theorem}
    \begin{itemize}[<+->]
        \item On each step $s \in \N$, for each $j \in \N$ we compute:
        \begin{itemize}
            \item A witness $w_{j,s}$ for $R_j$ at step $s$
            \item A restriction index $r_{j,s}$ to dictate the priority of the requirements
        \end{itemize}
        \item We'll enforce that $w_{j,s} = r_{j,s} = -1$ holds when no witness is known for $R_j$ at step $s$
        \item The indices that come before each $r_{j,s}$ are considered to be \textit{safe}, i.e. they don't break any requirement.
        \item $R_j$ gets \textit{injured} at step $s$ if $A_{s+1} - A_s$ contains some $x \leq r_{j,s}$.
    \end{itemize}
\end{frame}

\begin{frame}{The finite injury priority method}
\framesubtitle{The Friedberg-Muchnik theorem}
    \begin{itemize}[<+->]
        \item Let $A_0 = B_0 = \varnothing$ and let $w_{j,0} = r_{j,0} = -1$ for each $j \in \N$.
        \item Consider a generic step $s \in \N$.
        \begin{itemize}
            \item Assume $s = 2i$.
        \end{itemize}
        \item If $w_{2i,s} \neq -1$ we propagate the previous values because $R_{2i}$ already has a witness.
        \begin{itemize}
            \item We set $A_{s+1} = A_s$, $B_{s+1} = B_s$, $w_{2i,s+1} = w_{2i,s}$ and $r_{2i,s+1} = r_{2i,s}$
        \end{itemize} 
    \end{itemize}
\end{frame}

\begin{frame}{The finite injury priority method}
\framesubtitle{The Friedberg-Muchnik theorem}
    \begin{itemize}[<+->]
        \item Assume that $w_{2i,s} = -1$. Let $x$ be the minimum index such that $x \notin A_s$ and such that for all $k < 2i$ it holds that $x > r_{k,s}$.
        \item If $\Phi_{i,s}^{B_s}(x) \uparrow$, we preserve that $w_{2i,s+1} = r_{2i,s+1} = -1$ and propagate $A_{s+1} = A_s$, $B_{s+1} = B_s$. This trivially satisfies the requirement $R_{2i}$.
        \item Otherwise, if $\Phi_{i,s}^{B_s}(x) \downarrow$, we set:
        \[\begin{array}{ccc}
            w_{2i,s+1} = x & \qquad & r_{2i,s+1} = \max(x, \omega_{i,s}^{B_s}(x))\\\\
            A_{s+1} = A_{s} \cup \{x\} & \qquad & B_{s+1} = B_s
        \end{array}\]
    \end{itemize}
\end{frame}

\begin{frame}{The finite injury priority method}
\framesubtitle{The Friedberg-Muchnik theorem}
    \textbf{Claim}: For each $j \in \N$ it holds that:
    \begin{enumerate}
        \item $R_j$ is injured a finite number of times
        \item There is a step $s_0$ such that for all $s \geq s_0$ and for all $k < j$ no new index is added to $A_s$
    \end{enumerate}

    \textit{Proof.}
    \begin{itemize}[<+->]
        \item The requirement $R_j$ is injured at step $s$ if some $x \leq r_{j,s}$ is added to $A_s$, which happens only when there is a $k < j$ that adds a new index to $A_s$, i.e. when
        \[w_{j,k} = -1 \quad \Phi_{j,s}^{B_s}(x) \downarrow \quad \phi_{j,s}^{B_s}(x) = 0\]
        \item In other words, statement (2) of the claim follows from statement (1) by construction.
    \end{itemize}
\end{frame}

\begin{frame}{The finite injury priority method}
\framesubtitle{The Friedberg-Muchnik theorem}
    \begin{itemize}[<+->]
        \item Let $g_{j,s} = \sum\limits_{\substack{k <\, j \; s.t. \\ w_{k,s} \neq -1}} 2^{-(k+1)}$ be the injure value of $R_j$ at step $s$
        \item When $R_j$ is injured at step $s$, the smallest value $k < j$ such that $w_{k,s} \neq w_{k,s+1}$ must satisfy $w_{k,s} = -1$ and $w_{k,s+1} \neq -1$ by construction. Thus:
        \[g_{j,s+1} - g_{j,s} \geq 2^{-(j+1)} - \sum_{k < h < j} 2^{-(h+1)} = 2^{-j}\]

        \item Hence, for each step $s$ we have that $0 \leq g_{j,s} \leq 1-2^{-j}$, concluding that $R_j$ can be injured at most $2^{j}-1$ times
    \end{itemize}
\end{frame}

\begin{frame}{The finite injury priority method}
\framesubtitle{The Friedberg-Muchnik theorem}

    \begin{itemize}
        \item We observe that there is always a large enough step $s_0^*$ satisfying statements (1) and (2) of the Claim.
        
        \item This step guarantees that for each $j \in \N$ it holds that $\lim\limits_{s \to +\infty} w_{j,s}$ and $\lim\limits_{s \to +\infty} r_{j,s}$ exist and are finite.
        \begin{itemize}
            \item If for some $s' \geq s_0^*$ it holds that $w_{j,s'} \neq -1$ then it will hold from $s'$ onward
            \item Otherwise, there is a minimum value $x_s^*$ such that $x_s^* \notin A_s$ and such that $x_s^* > r_{k,s}$ for all $k < j$
            \item By construction $\Phi_{j,s}^{B_s}(x) \uparrow$ holds in this case
        \end{itemize}
    \end{itemize}

\end{frame}


\begin{frame}{The finite injury priority method}
\framesubtitle{The Friedberg-Muchnik theorem}

    \begin{itemize}[<+->]
        \item This concludes that, eventually, each requirement $R_j$ will be satisfied by the construction.
        \item Thus, $A = \bigcup\limits_{i \in \N} A_i$ and $B = \bigcup\limits_{i \in \N} B_i$ are such that $\Phi_s^A(x) \neq B$ and $\Phi_s^B(x) \neq A$ for all $s \in \N$
        \item Moreover, the use of the restriction values $r_{j,s}$ allows us to recursively enumerate the sets $A$ and $B$ by restricting our interest to the indexes between $0$ and $r_{j,s}$ for each $j \in \N$
        \begin{itemize}
            \item This implies that $A$ and $B$ are both r.e.!
        \end{itemize}
    \end{itemize}

    $\hfill \square$

\end{frame}

\begin{frame}{The finite injury priority method}
\framesubtitle{Other results}
    \begin{itemize}[<+->]
        \item \textbf{Infinite Injury Priority Argument}: \textcite{sacks} proved that the above construction can be extended to a countably infinite argument
        
        \item \textbf{Density of r.e. sets}: For each pair of r.e. sets $A, B$ there is another r.e. set such that $A <_T C <_T B$
        \item \textcite{simpson} proved that the first-order theory of $\mathcal{D}$ over the language $(\preceq, = )$ is many-one equivalent to the theory of true second-order arithmetic. 
    \end{itemize}
\end{frame}

\backmatter[notitle]
    
\begin{frame}{Bibliography}
\framesubtitle{{}}
    \printbibliography
\end{frame}

\end{document}
