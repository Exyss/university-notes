\documentclass[12pt,a4paper]{report}
\usepackage[affil-it]{authblk}
\usepackage{amsthm}
\usepackage{amssymb}
\usepackage{amsmath}
\usepackage{listings}
\usepackage{graphicx}
\usepackage{pgfplots}
\usepackage{float}
\usepackage{xcolor}
\usepackage{hyperref}
\usepackage{algpseudocode}
\usepackage[nameinlink]{cleveref}

\usepackage[
  backend=biber,
  style=alphabetic,
  sorting=anyt,
  minnames=3,
  minalphanames=3
]{biblatex}

\hypersetup{
    colorlinks=true,
    citecolor=blue,
    linkcolor=blue,
    urlcolor=blue,
    pdftitle={MLCS Homework 1 2024-25},
}

\newtheorem{question}{Question}
\theoremstyle{definition}
\newtheorem{solution}{Solution}

\definecolor{sapRed}{HTML}{6f0a19}
\definecolor{sapBlue}{HTML}{006778}

\newcommand{\curlyquotes}[1]{\textquotedblleft #1\textquotedblright}
\newcommand{\abs}[1]{\left|#1\right|}
\newcommand{\abk}[1]{\left\langle#1\right\rangle}
\newcommand{\N}{\mathbb{N}}                     % Natural Numbers
\newcommand{\Z}{\mathbb{Z}}                     % Integer Numbers

\newcommand{\model}[1]{\mathfrak{#1}}           % Gothic font

\addbibresource{./references.bib}

\begin{document}

    \setlength{\parskip}{5pt}               % Vertical spacing between paragraphs
    \setlength{\parindent}{0pt}             % Vertical spacing between paragraphs

    \title{Mathematical Logic in Computer Science \\ Homework 1 2024-25}
    \author{Simone Bianco, 1986936}
    \affil{Sapienza Università di Roma, Italy}
    \date{\today}

    \maketitle

    \begin{question}[Compactness]
        Let $\mathcal{L} = \{E(x, y)\}$ be the language of graphs.
        \begin{enumerate}
            \item For each fixed $n \in \N$ write a sentence $C_n$ such that for any graph $\mathcal{G}$, $\mathcal{G} \models C_n$ if and only if $\mathcal{G}$ contains an $n$-clique.
            \item Prove using Compactness that the property of being finitely colorable is not expressible by a theory in $\mathcal{L}$ over the class of graphs.
        \end{enumerate}
    \end{question}

    \begin{proof}[Solution to Question 1.]
        For any $n \in \N$, consider the following sentence $C_n$:
        \[C_n \equiv \exists x_1 \ldots \exists x_n \; \bigwedge_{i = 1}^n \bigwedge_{\substack{j = 1 \\i \neq j}}^n (E(x_i, x_j) \land \lnot (x_i = x_j))\] 

        \textbf{Claim 1.1}: $\mathcal{G} \models C_n$ if and only if $\mathcal{G}$ contains an $n$-clique

        \begin{proof}[Proof of Claim 1.1.]
            Suppose that $\mathcal{G} \models C_n$. Let $v_1, \ldots, v_n$ be the variables forming the assignment that satisfies $C_n$, i.e. $\left ( \begin{matrix}
                x_1 & \cdots & x_n \\ v_1 & \cdots & v_n
            \end{matrix} \right )$. Then, $\{v_1, \ldots,$ $v_n\}$ form an $n$-clique since every pair of vertices is adjacent to each other and they are distinct. Vice versa, suppose that $\mathcal{G}$ contains an $n$-clique $\{u_1, \ldots, u_n\}$. Then, the assignment formed by $\left ( \begin{matrix}
                x_1 & \cdots & x_n \\ u_1 & \cdots & u_n 
            \end{matrix}\right )$ satisfies the non-quantified formula inside $C_n$, concluding that $\mathcal{G} \models C_n$.
        \end{proof}

        Consider now the theory $T = \{C_n \mid n \in \N\}$, expressing the property of containing an $n$-clique for any $n \in \N$. By way of contradiction, suppose that there is a theory $T'$ expressing the property of being finitely colorable.

        \textbf{Claim 1.2}: $T^* = T \cup T'$ is finitely satisfiable.
        
        \begin{proof}[Proof of Claim 1.2.]
            Fix $S \underset{fin}{\subseteq} T^*$ and consider the set $S - T' = \{C_{i_1}, \ldots, C_{i_\ell}\}$. Let $M = \max(i_1, \ldots, i_\ell)$. Then, the complete graph $\mathcal{K}_M$ on $M$ vertices is $M$-colorable, hence finitely colorable, and contains an $M$-clique, which implies that it also contains an $h$-clique for any $h \leq M$. Thus, $\mathcal{K}_M \models S-T'$ and $\mathcal{K}_M \models T'$. Since it satisfies $T'$, it also satisfies any subset of $T'$ that may be inside $S$, concluding that $\mathcal{K}_M \models S$. 
        \end{proof}
    
        It's easy to see that if $\mathcal{G}$ is $k$-colorable then it doesn't contain a $k+1$ clique, concluding that $T^*$ is unsatisfiable. However, by Compactness we know that $T^*$ is finitely satisfiable if and only if it is satisfiable, raising a contradiction. Thus, the only possibility is that $T'$ doesn't exist.
    \end{proof}

    \newpage

    \begin{question}[Games and non-expressibility]
        Consider the structures $\model{A} = (\N, \leq)$ and $\model{B} = (\Z, \leq)$:
        \begin{enumerate}
            \item What is the smallest $k$ such that $\model{A} \equiv_k \model{B}$ does not hold?
            \item Prove that the Duplicator wins every finite moves game if the Spoiler is forced to always play in the same structure (i.e. either always in $\model{A}$ or always in $\model{B}$)
        \end{enumerate}
    \end{question}

    \begin{proof}[Solution to Question 2.]
        \quad

        \begin{enumerate}
            \item Through Ehrenfeucht's theorem, we know that $\model{A} \equiv_k \model{B}$ if and only if the Duplicator has a strategy to win the $k$-round game on $\model{A}, \model{B}$. It's easy to see that the Spoiler has a trivial strategy to win the $2$-round game. On the first round, the Spoiler picks $a_0 = 0_\N \in \N$ and the Duplicator answers with $b_0 \in \Z$. On the second round, the Spoiler can pick any value $b_1 \in \Z$ such that $b_1 < b_0$ in order to win the game since $0_\N$ is the minimal element in $\model{A}$. In fact, the query $\forall x \exists y \; (y \leq x) \land \lnot(y = x)$ is enough to distinguish the two structures, hence they cannot be $2$-equivalent. Moreover, the Duplicator can always easily win the $1$-game by replying with any value to the first choice of the Spoiler. Hence, we conclude that $2$ is the smallest value $k$ such that $\model{A} \equiv_k \model{B}$ doesn't hold.
            
            \item Consider the case where the Spoiler is forced to always play in $\model{A}$. Then, the Duplicator has a trivial strategy to win the $\infty$-game since $\Z$ contains $\N$: every choice $a_i \in \N$ is mapped to $b_i = a_i$ in $\Z$, preserving the order relations. Hence, the Duplicator can also win the $k$-game for any $k$.
            
            In the other case, instead, the Duplicator cannot win the $\infty$-game since the Spoiler has a strategy to win. On the first round, the Spoiler picks $b_0 = 0_\Z$ and the Duplicator answers with $a_0 \in \N$. Then, for any other $i$-th round, the Spoiler always picks $b_{i} = b_{i-1}-1 \in \Z$. Since there are a finite amount of elements in $\{0, \ldots, a_0\} \subset \N$, the Duplicator will eventually run out of options. Nonetheless, the Duplicator still has a strategy to win the $k$-game for any $k$. The idea is to map all the elements $z \in \Z$ with $z \geq 0$ to all the $n \in \N$ with $n \geq 2^k$, while reserving all the $n' \in \N$ with $n' < 2^k$ as a way to \curlyquotes{simulate density up to $k$ elements} in order to survive at least $k$ queries that ask for a middle element.

            Fix $k \in \N$. Let $a_{-1} = 2^k$ and $b_{-1} = 0$. On each round $i$, let $a_1, \ldots, a_{i-1} \in \N$ and $b_1, \ldots, b_{i-1} \in \Z$ be the choices made in the previous rounds. Let $A_i = \{a_j \mid -1 \leq j < i \text{ and } a_j \leq 2^k\}$ and let $B_i = \{b_j \mid -1 \leq j < i \text{ and } b_j \leq 0\}$. The Duplicator plays the $i$-th round through the following decision process:
            \begin{enumerate}
                \item If the Spoiler picks $b_i \in \Z$ such that $b_i \geq 0$ then the Duplicator answers with $a_i \in \N$ such that $a_i = 2^k + b_i$.
                \item If the Spoiler picks $b_i \in \Z$ such that $b_i < 0$ then:
                \begin{enumerate}
                    \item If $b_i < \min(B_{i-1})$ then the Duplicator answers with $a_i = \frac{1}{2} \min(A_{i-1})$
                    \item If $b_i > \min(B_{i-1})$ then the Duplicator answers with $a_i = \frac{1}{2}(a_j + a_t)$, where $j,t$ are two indices such that $b_j,b_t$ are the two values in $B_{i-1}$ that minimize $\abs{b_j - b_t}$ and such that $b_j < b_i < b_t$.
                \end{enumerate}
            \end{enumerate}

            \textbf{Claim 2.1}: For all $i \in \N$, if $i \leq k$ it holds that:
            \begin{enumerate}
                \item The distance between each pair $x,y \in A_i \cup \{0\}$ is at least $2^{k-i}$.
                \item $a_1, \ldots, a_i \mapsto b_1, \ldots, b_i$ is a partial isomorphism.
            \end{enumerate}

            \begin{proof}[Proof of Claim 2.1]
                Any time the choice made by the Spoiler falls in Case (a) the partial isomorphism is trivially preserved since each positive the integer gets only shifted by $2^k$ places. This allows us to restrict our focus entirely on choices of Case (b).
                
                We proceed by induction on $i$. When $i = 1$, we have that $A_0 = \{2^k\}$ and $B_0 = \{0\}$. Hence, since $b_1 < 0 = \min(A_0)$, the Duplicator answers with $a_1 = 2^k-1$. Since $\abs{a_{-1} - a_1} = \abs{0 - a_1} = 2^{k-1}$ and relations are preserved since no other choices have been made, the claim holds.
                
                Assume now that the claim holds for the $i$-th round. For the $(i+1)$-th round, we consider the two subcases:
                \begin{enumerate}
                    \item If $b_{i+1} < \min(B_i)$ then by setting $a_{i+1} = \frac{1}{2}\min(A_i)$ we guarantee that $b_{i+1}, a_{i+1}$ become the new minimal elements in both orders, preserving relations. By inductive hypothesis, we know that $\abs{0 - \min(A_i)} \geq 2^{k-i}$, hence $\abs{0 - a_{i+1}} \geq 2^{k-(i+1)}$, while we also get that $\abs{a_j - a_{i+1}} \geq 2^{k-i} \geq 2^{k-(i+1)}$ for all $a_j \in A_i$ since $a_{i+1} < \min(A_i)$.
                    \item If $b_{i+1} > \min(B_i)$ then the elements $b_j,b_t$ are guaranteed to exist since $0 > b_{i+1} > \min(B_i)$. The minimality of the distance between $a_j$ and $a_t$ allows us to restrict our interest to proving that the distance claim holds on the latter elements. By inductive hypothesis, we know that $\abs{a_j - a_t} \geq 2^{k-i}$. By assigning $a_{i+1}$ to the element lying exactly in the middle between these two elements, we get that $\abs{a_j - a_{i+1}} \geq 2^{k-(i+1)}$ and $\abs{a_{i+1} - a_t} \geq 2^{k-(i+1)}$. This choice also preserves the order relation.
                \end{enumerate}
            \end{proof}

            The distance property between each element chosen in $\N$ guarantees that the elements selected by the Duplicator are all distinct for at least $k$ rounds (to be precise,they can survive for $k+1$ rounds), even if the choice of the Spoiler always falls in the second case.
        \end{enumerate}

    \end{proof}


\end{document}